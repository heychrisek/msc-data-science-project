% see /usr/local/texlive/2020/texmf-dist/doc/latex/lshort-english/lshort.pdf for textbook!


\documentclass[11pt, oneside]{article}   	% use "amsart" instead of "article" for AMSLaTeX format
\usepackage{geometry}                		% See geometry.pdf to learn the layout options. There are lots.
\geometry{letterpaper}                   		% ... or a4paper or a5paper or ... 
%\geometry{landscape}                		% Activate for rotated page geometry
%\usepackage[parfill]{parskip}    		% Activate to begin paragraphs with an empty line rather than an indent
\usepackage{graphicx}				% Use pdf, png, jpg, or eps§ with pdflatex; use eps in DVI mode
								% TeX will automatically convert eps --> pdf in pdflatex		
\usepackage{amssymb}

%SetFonts

%SetFonts


\title{Brief Article}
\author{The Author}
%\date{}							% Activate to display a given date or no date

\begin{document}
\maketitle
%\section{}
%\subsection{}

\begin{abstract}
The abstract abstract. The abstract abstract.
\end{abstract}

words words words\footnote{here is a footnote about words}

``Please press the `x' key.''

Here's a quote:

\begin{quote}
To be or not to be...skaespeare quote etc
\end{quote}

\tableofcontents

\section{Start of a section}
Section Section Section
\subsection{Start of a subsection}
Sub Sub Sub Sub
\subsubsection{Start of a subsubsection}
\paragraph{Start of a paragraph}
\subparagraph{Start of a subparagraph}


\paragraph{Python code paragraph\newline}

Some code:

\begin{verbatim}
// python
def something(arg):
    return arg+1
\end{verbatim}

\newpage

% Example 1
     \ldots when Einstein introduced his formula
     \begin{equation}
e = m \cdot c^2 \; ,
\end{equation}
which is at the same time the most widely known and the least well understood physical formula.\\

% Example 2
     \ldots from which follows Kirchhoff's current law:
     \begin{equation}
\sum_{k=1}^{n} I_k = 0 \; . \end{equation}
     Kirchhoff's voltage law can be derived \ldots\\

% Example 3
\ldots which has several advantages.
\begin{equation} I_D = I_F - I_R
\end{equation}
is the core of a very different transistor model. \ldots

\begin{equation}1+1 = 1\end{equation}

\section{Start of the NEXT section}
\subsection{Start of the NEXT subsection}
\subsubsection{Start of the NEXT subsubsection}
\paragraph{Start of the NEXT paragraph}
\subparagraph{Start of the NEXT subparagraph}

% starred section so it doesn't appear in table of contents
\section*{Appendix}
 
\section{References}

\end{document}  
